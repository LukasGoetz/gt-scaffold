\documentclass[xcolor=pst]{beamer}
\usepackage[utf8]{inputenc}
\usepackage{ngerman}
\usepackage{beamerthemesplit}
\usepackage{epsfig}
\usepackage{tikz}
\usepackage{siunitx}
\usepackage{pbox}
\usepackage{pdfpages}
\usepackage{verbatim}
\usepackage{units}
\usepackage{algpseudocode}

\usepackage{color}
\usetheme{Antibes}
%\usepackage{multirow}

\usetikzlibrary{shapes.geometric, calc}

% fusszeile so aufteilen, dass autoren und titel reinpassen und seitenzahl hinzu
\setbeamertemplate{footline}
{
  \leavevmode%
  \hbox{%
  \begin{beamercolorbox}[wd=.25\paperwidth,ht=2.25ex,dp=1ex,center]{author in head/foot}%
    \usebeamerfont{author in head/foot}\insertshortauthor
  \end{beamercolorbox}%
  \begin{beamercolorbox}[wd=.75\paperwidth,ht=2.25ex,dp=1ex,center]{title in head/foot}%
    \usebeamerfont{title in head/foot}\insertshorttitle\hspace*{2.5em}\insertframenumber
  \end{beamercolorbox}}%
  \vskip0pt%
}


% so definiert man neue makros
\newcommand{\IFF}{\Leftrightarrow}
\newcommand{\todo}[1]{\textbf{\color{red}todo:\color{black}#1}}

\author{
  Lukas Götz, Stefan Dang \& Dorle Osterode
}
\title{Gt-Scaffolder: TODO}
\institute[FBI - UniHH]{Universität Hamburg - Fachbereich Bioinformatik}
\date{2015-01-30}

\subject{}
\keywords{}

\begin{document}
\begin{frame}[plain]
\titlepage
\end{frame}

% keine seitenzahl auf der inhaltsangabe zeigen (theme nur fuer diese folie anpassen)
\bgroup
\makeatletter
\setbeamertemplate{footline}
{
  \leavevmode%
  \hbox{%
  \begin{beamercolorbox}[wd=.25\paperwidth,ht=2.25ex,dp=1ex,center]{author in head/foot}%
    \usebeamerfont{author in head/foot}\insertshortauthor
  \end{beamercolorbox}%
  \begin{beamercolorbox}[wd=.75\paperwidth,ht=2.25ex,dp=1ex,center]{title in head/foot}%
    \usebeamerfont{title in head/foot}\insertshorttitle\hspace*{2.5em}
  \end{beamercolorbox}}%
  \vskip0pt%
}
\makeatother
\begin{frame}{Übersicht}
\tableofcontents
\end{frame}
\egroup % ab hier das normale theme weiterverwenden


\section{Motivation}

\begin{frame}
\setcounter{framenumber}{1}
  \frametitle{Das Assembly-Problem}
\end{frame}

\section{Methoden}
\begin{frame}
  \frametitle{Scaffolding}
\end{frame}

\section{Ergebnisse}
\begin{frame}
  \frametitle{Vergleich mit SGA}

\end{frame}

\section{Diskussion und Ausblick}
\begin{frame}
  \frametitle{Diskussion}
  \begin{itemize}
  \item weniger Abhängigkeiten als SGA
  \item 
  \end{itemize}
\end{frame}

\begin{frame}
  \frametitle{Ausblick}
  \begin{itemize}
  \item was wir noch alles machen muessen (kann erst am ende
    ausgefuellt werden)
  \end{itemize}
\end{frame}
\end{document}
