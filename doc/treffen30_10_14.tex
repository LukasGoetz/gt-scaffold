\documentclass[a4paper,10pt,parskip]{scrartcl}
\usepackage[utf8]{inputenc}
\usepackage[ngerman]{babel}
\usepackage[T1]{fontenc}
\usepackage{lmodern}
\usepackage{amsmath}
\usepackage{amsfonts}
\usepackage{amssymb}
\usepackage{graphicx}
\usepackage{placeins}
\usepackage[lined,algoruled,linesnumbered]{algorithm2e}


\title{Treffen 30.10.2014}
\author{Dorle Osterode, Stefan Dang \& Lukas Götz}

\begin{document}

\maketitle{}

Anhand der verwendeten Datentypen in der Implementation des
Scaffoldings in dem Programm SGA, wurde eine Liste an grundlegend
benötigten Datentypen mit den jeweils benötigten Informationen
bzw. Feldern erstellt. Die Liste orientiert sich sehr stark an der
Implementation.

\begin{itemize}
\item Datentyp Graph: Der Graph enthält die Contigs als Knoten und
  Kanten zwischen zwei Knoten, wenn diese Knoten über eine
  \textit{paired-end} oder \textit{mate-pair} Information miteinander
  verbunden sind. Auf diesem Graph sollen dann die Algorithmen
  arbeiten. Der Graph ist bidirektional. Der Graph enthält folgende
  Informationen:
  \begin{itemize}
  \item Knoten
  \item Kanten
  \item ungewurzelt
  \end{itemize}
\item Datentyp Knoten: Ein Knoten ist spezialisiert und enthält alle
  Informationen über den zu repräsentierenden Contig direkt. Diese
  Informationen könnten allerdings auch in einen weiteren Datentyp
  ausgelagert werden. Der Vorteil einer Auslagerung wäre, dass eine
  generische Graph-Struktur und damit generische Graph-Algorithmen mit
  \textit{call-back}-Funktionen verwendet werden können, falls eine
  solche Graph-Struktur in den Genometools schon vorhanden ist. Ein
  Knoten bzw. Contig braucht folgende Informationen:
  \begin{itemize}
  \item VertexID m-id \# eindeutige ID für den Knoten
  \item size-t m-seqLen \# Länge des Sequenz, die der Contig darstellt
  \item double m-AStatistic \# Wert der A-Statistik, um Contigs als
    REPEAT oder UNIQUE klassifizieren zu können
  \item double m-estCopyNumber \# abgeschätzte Anzahl an Vorkommen des
    Contigs im Genom
  \item ScaffoldEdgePtrVector m-edges \# Sammlung von Kanten, die von
    dem Contig abgehen
  \item ScaffoldVertexClassification m-classification \#
    Klassifikation: REPEAT oder UNIQUE
  \item GraphColor m-color \# Markierung für Algorithmen
  \item bool m-hasConflictingLink \# $\top$, wenn der Contig im
    Konflikt stehende Kanten hat
  \end{itemize}
\item Datentyp Kante: Eine Kante verbindet zwei Knoten und hat eine
  Richtung. Eine Kante braucht folgende Informationen:
  \begin{itemize}
  \item ScaffoldVertex* m-pEnd \# der Knoten, zu dem die Kante führt
  \item ScaffoldEdge* m-pTwin \# Kante, die genau in die andere
    Richtung führt
  \item ScaffoldLink m-link \# Informationen zu der Verbindung
    zwischen den Knoten
  \item GraphColor m-color \# Markierung für Algorithmen
  \end{itemize}
\item Datentyp Link: Ein Link enthält alle Informationen zu einer
  Verbindung zwischen zwei Knoten. Ein Link hat folgende Information
  über eine Kante:
  \begin{itemize}
  \item std::string endpointID \# Id von dem Knoten auf den die Kante
    zeigt
  \item int seqLen \# Länge der Sequenz des Contigs, auf den die Kante
    zeigt
  \item int distance \# Abschätzung der Entfernung der verbundenen
    Contigs
  \item double stdDev \# Standardabweichung von der abgeschätzten
    Entfernung
  \item int numPairs \# Anzahl der Distanzinformationen, die ein
    Anzeichen für die Verbindung der Contigs geben
  \item ScaffoldLinkType type \# Typ der Verbindung: Distanz aus einer
    Eingabedatei (SLT-DISTANCEEST), (SLT-REFERENCE), (SLT-INFERRED),
    eine andere Verbindung (SLT-NOTYPE)
  \item EdgeData edgeData \# enthält die Richtung (Sense, Antisense)
    und welche Stränge die paired-Information enthalten (die gleiche
    Richtung oder das Reverse)
  \end{itemize}
\end{itemize}



\begin{algorithm}[H]
  \SetAlgoLined
  \KwData{Contigs, Distanzinformationen}
  \KwResult{Scaffold der einzelnen Contigs}
  Graph konstruieren mit Contigs als Knoten und Kanten 
  aus den Distanzinformationen\;
  repititive Knoten anhand von Astatistik entfernen\;
  \dots alle möglichen Dinge herausfiltern, wenn gewünscht (nicht default)\;
  Polymorphe Knoten entfernen und inkonsistente Kanten löschen\;
  Anfangs- und Endknoten von Zyklen löschen\;
  Scaffold aus dem gefilterten Graphen berechnen\;
  \caption{Grober Ablauf des Scaffoldings bei SGA}
\end{algorithm}

\begin{algorithm}[H]
  \SetAlgoLined
  \KwData{AStatistik-Datei}
  \KwResult{Graph ohne Knoten mit zu kleinem AStatistik-Wert oder zu großem 
    estCopyNum-Wert}
  Lade AStatistik- und estCopyNum-Werte für jeden Knoten 
  aus einer Datei und setzte die Werte für jeden Knoten \;
  \ForEach{Knoten im Graph}{
    \If{Knoten ist REPEAT}{continue\;}
    \eIf{Astatistik $>$ min-Astatistik}
        {Knoten ist UNIQUE\;}
        {Knoten ist REPEAT\;}
    \If{estCopyNum $>$ min-estCopyNum}{Knoten ist REPEAT\;}
  }
  Lösche alle Knoten und deren Kanten, die als REPEAT gekennzeichnet sind.
  \caption{Astatistik behandeln (Schritt 2)}
\end{algorithm}

\begin{algorithm}[H]
  \KwData{Graph}
  \KwResult{Graph ohne polymorphe Knoten}
  \While{es gibt noch polymorphe Knoten}{
    \ForEach{Knoten}{
      \If{Knoten ist REPEAT}{breche das ganze ab\;}
      \If{jedes Paar von Kanten in die gleiche Richtung mit mehr 
        als der Wahrscheinlichkeit p-cutoff uneindeutig geordnet
        ist {\bf and} die Summe der estCopyNum kleiner als cn-cutoff ist}
         {Knoten wird als polymorph markiert.\;}
      }
  }
  Lösche alle polymorph markierten Knoten\;
  \caption{Polymorphe Knoten entfernen (Schritt 4a)}
\end{algorithm}

\begin{algorithm}[H]
  \KwData{Graph}
  \KwResult{Graph ohne inkonsistente Kanten}
  \ForEach{Knoten}{
    berechne den längsten Overlap aller Paare von Kanten in die gleiche Richtung\;
    \If{längster Overlap $>$ 400}{
      färbe alle Kanten und Rückkanten schwarz\;
    }
    }
  Lösche alle schwarzen Kanten\;
  \caption{Inkonsistente Kanten entfernen (Schritt 4b)}
\end{algorithm}

\begin{algorithm}[H]
  \KwData{Graph}
  \KwResult{Graph ohne Zyklen}
  \While{noch nicht fertig}{
  Suche alle Zusammenhangskomponenten\;
  \ForEach{Zusammenhangskomponente}{
    Suche alle terminalen Knoten\;
    \ForEach{terminalen Knoten}{
      Suche mit einer Tiefensuche nach Rückkanten zu diesem Knoten\;
      \If{Rückkante gefunden}{
        markiere den terminalen Knoten und den Knoten, von dem die 
        Rückkante ausgeht, als REPEAT\;
        }
      }
    }
  Lösche alle Knoten, die als REPEAT markiert sind\;
  }
  \caption{Löse Zyklen auf (Schritt 5)}
\end{algorithm}

\begin{algorithm}[H]
  Färbe alle Kanten schwarz\;
  Berechne alle Zusammenhangskomponenten\;
  \ForEach{Zusammenhangskomponente}{
    Berechne alle terminalen Knoten\;
    \ForEach{terminalen Knoten}{
      Berechne einen Walk durch die Zusammenhangskomponente 
      mit der meisten abgedeckten Sequenz und markiere diesen 
      Walk weiß um ihn zu behalten\;
      }
    }
  Lösche alle schwarzen Kanten\; 
  \caption{Scaffold berechnen (Schritt 6)}
\end{algorithm}

\end{document}
